\documentclass[10pt]{article}
\usepackage[final]{graphicx}
\usepackage{amsfonts}

\topmargin-.5in
\textwidth6.6in
\textheight9in
\oddsidemargin0in

\def\ds{\displaystyle}
\def\d{\partial}

\begin{document}

\centerline{\large \bf Identifying Precision Treatment for Rheumatoid Arthritis with Reinforcement Learning}

\vspace{.1truein}

\def\thefootnote{\arabic{footnote}}
\begin{center}
  Chixiang Chen\footnote{Biostatistics, Pennsylvania State University},
  Ashley Gannon\footnote{Scientific Computing, Florida State University},
  Duwani Katumullage\footnote{Statistics, Sam Houston State University},
  Miaoqi Li\footnote{Department, University},
  Mengfei Liu\footnote{Department, University of Cincinnati}
  Rebecca North\footnote{Statistics, NC State University}
  Jialu Wang\footnote{Statistics, College of William and Mary}
\end{center}

%\vspace{.1truein}

\begin{center}
Faculty Mentors: Grant Weller, Victoria Mansfield, Yingong Guo\footnote{Savvysherpa},
Daniel Luckett \footnote{UNC}
\end{center}

\vspace{.3truein}
\centerline{\bf Abstract}


Rheumatoid Arthritis (RA) is an autoimmune condition in which a patient's own immune system attacks the lining of joints, leading to painful swelling and aches and, eventually, long-term damage from bone erosion and joint deformity. Management of RA is challenging for today's healthcare system due to a lack of clinically validated measures of disease activity and progression, a high rate of comorbid conditions, and a proliferation of prescription drug therapies with unclear guidance on optimal therapy for a given patient. In pursuit of the identification of optimal dynamic treatment regimes for RA patients, we explore the empirical treatment regimes observed in longitudinal administrative health data on a cohort of over 6,500 RA patients and propose a framework to identify an optimal regime.

\begin{itemize}
\item Summarize the results presented in the report, and the contributions
of your research.

\item Readers should not have to look at the rest of the paper in order to 
understand the abstract.

\item Keep it short and to the point.
\end{itemize}

\section{Introduction}
It should be written as much as possible in non-technical terms, so that a
lay reader can understand the context and the contribution of the paper.

\begin{itemize}
\item Describe the problem you are trying to solve, the approach
you took, and summarize your contribution and results.

\item Review the history of this problem, and existing literature.

\item Give an outline of the rest of the paper.
\end{itemize}

\section{The Problem}
\begin{itemize}
\item Give a precise technical description of your problem. 

\item State and justify all your assumptions. 

\item Define notation. 

\item Describe your data, how you collected them, their properties,
and whether you did 
anything to them (removed noise, filled in missing data, 
applied normalizations).
\end{itemize}

\section{The Approach}

\begin{tabular}{l|l}
\hline
 & member (N = 6846)\\
\hline
\bf{Age at First Diagnosis} & ~\\
\hline
~~ min & 12\\
\hline
~~ max & 64\\
\hline
~~ mean (sd) & 47.26 $\pm$ 10.99\\
\hline
~~ median (iqr) & 50.00 (41.00, 56.00)\\
\hline
\bf{Gender} & ~\\
\hline
~~ Male & 1,753 (26)\\
\hline
~~ Female & 5,093 (74)\\
\hline
\bf{AIDS/HIV} & ~\\
\hline
~~ Prior & 9 (0)\\
\hline
~~ After & 13 (0)\\
\hline
\bf{Acute Myocardial Infarction} & ~\\
\hline
~~ Prior & 67 (1)\\
\hline
~~ After & 81 (1)\\
\hline
\bf{Angina} & ~\\
\hline
~~ Prior & 409 (6)\\
\hline
~~ After & 417 (6)\\
\hline
\bf{Cancer} & ~\\
\hline
~~ Prior & 294 (4)\\
\hline
~~ After & 357 (5)\\
\hline
\bf{Cerebrovascular Disease} & ~\\
\hline
~~ Prior & 303 (4)\\
\hline
~~ After & 296 (4)\\
\hline
\bf{Congestive Heart Failure} & ~\\
\hline
~~ Prior & 131 (2)\\
\hline
~~ After & 192 (3)\\
\hline
\bf{COPD} & ~\\
\hline
~~ Prior & 1,423 (21)\\
\hline
~~ After & 1,296 (19)\\
\hline
\bf{Dementia} & ~\\
\hline
~~ Prior & 7 (0)\\
\hline
~~ After & 11 (0)\\
\hline
\bf{Diabetes} & ~\\
\hline
~~ Prior & 782 (11)\\
\hline
~~ After & 866 (13)\\
\hline
\bf{Hypertension} & ~\\
\hline
~~ Prior & 2,182 (32)\\
\hline
~~ After & 2,315 (34)\\
\hline
\bf{Liver Disease} & ~\\
\hline
~~ Prior & 433 (6)\\
\hline
~~ After & 557 (8)\\
\hline
\bf{Paralysis} & ~\\
\hline
~~ Prior & 34 (0)\\
\hline
~~ After & 32 (0)\\
\hline
\bf{Peripheral Vascular Disease} & ~\\
\hline
~~ Prior & 243 (4)\\
\hline
~~ After & 275 (4)\\
\hline
\bf{Renal Failure} & ~\\
\hline
~~ Prior & 90 (1)\\
\hline
~~ After & 192 (3)\\
\hline
\bf{Ulcers} & ~\\
\hline
~~ Prior & 119 (2)\\
\hline
~~ After & 97 (1)\\
\hline
\bf{Depression} & ~\\
\hline
~~ Prior & 1,276 (19)\\
\hline
~~ After & 1,129 (16)\\
\hline
\bf{Skin Ulcers} & ~\\
\hline
~~ Prior & 865 (13)\\
\hline
~~ After & 565 (8)\\
\hline
\end{tabular}

\begin{itemize}
\item Present and justify your approach for solving the problem. 
\item Explain the advantages of your approach over existing ones.

\item Tell a story.
Don't just say: ``I did this, then I did this, and at last I did this''.
\end{itemize}

\section{Computational Experiments}
Give enough details so that readers can duplicate your experiments.

\begin{itemize}
\item Describe the precise purpose of the experiments, and what they 
are supposed to show.

\item Describe and justify your test data, and any assumptions you made to 
simplify the problem.

\item Describe the software you used, and the 
parameter values you selected.

\item 
For every figure, describe the meaning and units of the coordinate axes, 
and what is being plotted.

\item Describe the conclusions you can draw from your experiments
\end{itemize}

\section{Summary and Future Work}
\begin{itemize}
\item Briefly summarize your contributions, and their possible
impact on the field (but don't just repeat the abstract or introduction).
\item Identify the limitations of your approach.
\item Suggest improvements for future work.
\item Outline open problems.
\end{itemize}



\end{document}